\documentclass[12pt]{article}
\usepackage[margin=1in]{geometry}
\usepackage{microtype}

\begin{document}
%\maketitle
\begin{center}
\Large{\textbf{Linux cheatsheet}}\\
\end{center}

\section{Directories (folders)}

\subsection{mkdir: make a directory}
To make a directory use 
\texttt{mkdir directoryName}

\subsection{pwd: make a directory}
To make a directory use 

\subsection{Special directories}
\begin{definition}
\item[\texttt{.}] The current ``working'' directory (often unnecessary to specify this). This is like opening a folder in Windows or Mac.
\item[\texttt{..}] The parent directory of the current directory e.g. if you are in \texttt{/var/root} then \texttt{..} is \texttt{/var}. This is like ``Up'' in Windows file explorer.
\item[\texttt{~}] Your home directory (different for each user). On this system, this is \texttt{/var/root}. This is like the ``My Documents'' directory in Windows or ``Home'' on Macs.
\item[\texttt{/}] The root directory. The very base of the filesystem.  Something like the ``Computer'' directory in Windows.
\end{definition}


\end{document}
