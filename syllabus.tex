
\documentclass[12pt]{article}
\usepackage[margin=1in]{geometry}
\usepackage{microtype}
\usepackage{fancyhdr} % This should be set AFTER setting up the page geometry
\usepackage{lastpage}
\usepackage{longtable}
\pagestyle{fancy} % options: empty , plain , fancy
\renewcommand{\headrulewidth}{0.4pt} % customise the layout...
\rhead{\footnotesize BIOL 567 syllabus page \thepage/\pageref{LastPage}}
\renewcommand\footrulewidth{0pt}
\usepackage[pdftitle={BIOL 567 syllabus},pdfauthor={Scott Sherrill-Mix},colorlinks=true,urlcolor=black,linkcolor=black]{hyperref} 
\title{BIOL 567}
\date{Fall 2017}
\author{Scott Sherrill-Mix}

\begin{document}
%\maketitle
\thispagestyle{plain}
\begin{center}
\Large{\textbf{BIOL 567\\ Biological data analysis}}\\
\Large{\textbf{Fall 2017}}
\end{center}

\noindent
\textbf{Instructor:} Scott Sherrill-Mix\\
\textbf{Course location:} 123 Someplace\\
\textbf{Course time:} Wed 1-3pm\\
\textbf{Course website:} \url{http://somewhere.edu/BIOL657/}\\

\noindent
\textbf{Office:} 345 Somewhere\\
\textbf{Office hours:} By appointment or Wed 3-5pm\\
\textbf{Email:} \href{mailto:shescott@upenn.edu}{shescott@upenn.edu}\\
\textbf{Phone:} (234) 567-7890

\section*{Course description}
This course aims to prepare students for data analysis in biological research. The primary objective is to enable students to incorporate bioinformatic, statistical and scripting tools into their research. The course is intended for students with little familiarity with programming but a desire to learn. A basic knowledge of cell biology is suggested. Class time will be split between lectures, demonstrations and in-class exercises. I think hands on application is the best way to learn these topics so there will be weekly assignments outside class intended to take 1-3 hours to complete and a final project. 

\section*{Equipment}
Each student will need a laptop (with internet connection and the ability to install free open-source software) for in class activities. Please try to avoid websites/media that is not class related while in the classroom. 

\section*{Book}
No textbook is required for this course. Reading assignments will be detailed in the weekly homeworks and sources of additional information will be given at the end of each lecture.

\section*{Objectives}
The goal of the course is for students to achieve a comfortable familiarity with:
\begin{itemize}
	\item Command line and text editors
	\item Source control
	\item Scripting and batch processing
	\item Reproducible analysis and plotting 
	\item Common bioinformatic algorithms and databases
\end{itemize}
I hope this will enable students to use these techniques in their own research and continue learning after leaving the class.

\section*{Grading}
\subsection*{Weekly assignments (40\%)}
There will be weekly assignments in applied data analysis. Some components of these may be used during in-class activities.
\subsection*{Final project (40\%)}
A final research project with written documentation and code (30\%) along with an in class presentation (10\%).
\subsection*{Final exam (10\%)}
This will be a short exam similar to a programming test one might encounter at a job interview.
\subsection*{Class participation (10\%)}
Participation entails class attendance, participation during in-class activities and mutual respect for other students and teachers.

\subsection*{Late grading}
Assignments are due at midnight on the due date. Late assignments with be accepted up to one week after the due date with a 30\% grade deduction. Each student will get one free late assignment. Final project write-ups will be accepted up to 3 days late with a cumulative 10\% deduction per day.

\subsection*{Plagiarism policy}
Students are allowed to work together and discuss assignments but each student is responsible that they independently complete the assignments and completely understand what is turned in. Direct reuse of another person's code in not acceptable in the weekly assignments. The point of the assignments is the journey not the destination. In the final project, code can be reused with correct licensing and documentation (\emph{all included code written by anyone other than the student must be explicitly documented even if not required by the software license}).

\section*{Schedule}
\newcounter{rownumbers}
\newcommand\rownumber{\stepcounter{rownumbers}\arabic{rownumbers}}
\newcounter{assnumbers}
\newcommand\assnumber{\stepcounter{assnumbers}\arabic{assnumbers}}
\begin{longtable}{|c|p{4in}|p{1.5in}|}
\hline
Week & Topics & Assignments due\\
\hline
\rownumber & Introduction to ssh and command line\newline Introduction to text editors\newline Getting help & ---\\
\hline
\rownumber & Bash scripting \newline Introduction to python\newline Programming basics (variables, data types)  & Assignment \assnumber\\
\hline
\rownumber & Creating functions \newline Benefits of .txt \newline Source control & Assignment \assnumber\\
\hline
\rownumber & Introduction to R\newline Plotting& Assignment \assnumber\\
\hline
\rownumber & Flow control and logic \newline Command line tools & Assignment \assnumber\\
\hline
\rownumber & Data structures overview \newline Lists and dictionaries & Assignment \assnumber\\
\hline
\rownumber & Best practices \newline Reuseable code \newline Reproducible research & Final project description\\
\hline
\rownumber & Python modules\newline scipy, multiprocessing, subprocess, biopython& Assignment \assnumber\\ 
\hline
\rownumber & String algorithms \newline Levenshtein distance& Assignment \assnumber\\
\hline
\rownumber & BLAST\newline Bio databases& Assignment \assnumber\\
\hline
\rownumber & Statistics in R& Assignment \assnumber\\
\hline
\rownumber & Final test& ---\\
\hline
\rownumber & Student presentations& Final report due\\
\hline
\end{longtable}
\end{document}
