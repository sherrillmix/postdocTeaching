
\documentclass[12pt]{article}
\usepackage[margin=1in]{geometry}
\usepackage{microtype}
\usepackage{fancyhdr} % This should be set AFTER setting up the page geometry
\usepackage{lastpage}
\pagestyle{fancy} % options: empty , plain , fancy
\renewcommand{\headrulewidth}{0.4pt} % customise the layout...
\rhead{\footnotesize BIOL 567 syllabus page \thepage/\pageref{LastPage}}
\renewcommand\footrulewidth{0pt}
\usepackage[pdftitle={BIOL 567 syllabus},pdfauthor={Scott Sherrill-Mix},colorlinks=true,urlcolor=black]{hyperref} 
\title{BIOL 567}
\date{Fall 2017}
\author{Scott Sherrill-Mix}

\begin{document}
%\maketitle
\thispagestyle{plain}
\begin{center}
\Large{\textbf{BIOL 567\\ Introduction to biological data analysis}}\\
\Large{\textbf{Fall 2017}}
\end{center}
\bigskip

\noindent
\textbf{Instructor:} Scott Sherrill-Mix\\
\textbf{Location:} 123 Someplace\\
\textbf{Time:} Wed 1-3pm\\
\textbf{Website:} \url{http://somewhere.edu/BIOL657/}\\
\bigskip

\noindent
\textbf{Office:} 345 Somewhere\\
\textbf{Office hours:} By appointment or Wed 3-5pm\\
\textbf{Email:} \href{mailto:shescott@upenn.edu}{shescott@upenn.edu}\\
\textbf{Phone:} (234) 567-7890

\section*{Course description}
Some more text and a lot more. The Department of Computational Medicine and Bioinformatics (DCMB) invites applications from outstanding candidates for a junior level research track faculty position. This is an exciting opportunity to jumpstart your career with high impact research projects and the potential to advance to an independent principal investigator at the University of Michigan. The position will involve integrative analyses of data from various high-throughput molecular approaches (epigenomics, regulomics, transcriptomics, and genomics) with a focus on projects related to cancer and environmental health, and will be supervised by Dr. Maureen Sartor. 

\section*{Objectives}
\begin{itemize}
	\item Batch process large datasets
	\item Common databases
	\item Source control
	\item Reproducible research
\end{itemize}

\section*{Grading}


\newcommand{\nextitem}{\par\hspace*{\labelsep}\textbullet\hspace*{\labelsep}}
\section*{Schedule}
\begin{table}[h!]
\normalsize
\begin{tabular}{ | c | p{.8\textwidth} | }
\hline
\textbf{Week} & \textbf{Content} \\
\hline
Week 1 &
	\nextitem Something interesting
	\nextitem Reading assignment: Something interesting \\
\hline
Week 2 &
	\nextitem Something interesting
	\nextitem Reading assignment: Something interesting \\
\hline
Week 3 &
	\nextitem Something interesting
	\nextitem Reading assignment: Something interesting \\
\hline
Week 4 &
	\nextitem Something interesting
	\nextitem Reading assignment: Something interesting \\
\hline
Week 5 &
	\nextitem Something interesting
	\nextitem Reading assignment: Something interesting \\
\hline
Week 6 &
	\nextitem Something interesting
	\nextitem Reading assignment: Something interesting \\
\hline
Week 7 &
	\nextitem Something interesting
	\nextitem Reading assignment: Something interesting \\
\hline
Week 8 &
	\nextitem Something interesting
	\nextitem Reading assignment: Something interesting \\
\hline
Week 9 &
	\nextitem Something interesting
	\nextitem Reading assignment: Something interesting \\
\hline
Week 10 &
	\nextitem Something interesting
	\nextitem Reading assignment: Something interesting \\
\hline
Week 11 &
	\nextitem Something interesting
	\nextitem Reading assignment: Something interesting \\
\hline
Week 12 &
	\nextitem Something interesting
	\nextitem Reading assignment: Something interesting \\
\hline
Week 13 &
	\nextitem Something interesting
	\nextitem Reading assignment: Something interesting \\
\hline
Week 14 &
	\nextitem Something interesting
	\nextitem Reading assignment: Something interesting \\
\hline
\end{tabular} 
\end{table}

\section*{Course policies}

\end{document}
